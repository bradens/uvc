\documentclass{article}
\usepackage[utf8]{inputenc}

\title{Enterprise Security Architecture\\Case Study : Central Credit Union}
\author{Braden Simpson\\braden@uvic.ca\\V00685500}
\date{January 7, 2013}

\begin{document}

\maketitle

\section{Introduction}
The Central Credit Union (CCU) is a centre for providing secure services for banking to all of the credit unions of Alberta and Manitoba.  It offers internet banking, automated banking, chequing services, real time banking and more.  The Central Credit Union is owned by all the credit unions themselves, and exists to offer large enterprise services to small credit unions.  In this paper I will explain the proposed Enterprise Security Architecture (ESA) for the CCU in depth.

\section{Organizational Structure}
The organizational structure of an ESA is key and has to be carefully constructed. In order to have an ESA, there must be a committee, comprised of higher level executives, which handles the major decisions, a team of lower level managers, which carry out the actual work of the decisions made by the committee.  In the organizational structure shown in the case study\footnote{GrantThornton -- http://www.ece.uvic.ca/~henrylee/2013/01/01ESA-Case.pdf} there is a C-Level group of executives, which are included in the committee, including the Corporate security manager Linda, which could act as the Security Officer.  Alternatively, another option would be to hire a new person to act as the Security Officer, whos job it is to sit in at both the committee meetings and the team meetings.  She chairs them both and ensures the decisions are translated correctly into the managers meetings. 

\section{Key Components}
The ESAs key components include policies, standards, training, organizational standards, and more.  The policies have to be realisitic in their expectations, detailed, and concise.  In order for the company-wide acceptance, the policies and standards have to be simple enough for all types of employees, for example, a password policy that is most secure might be that each employee needs a 40-character password -- this is obviously impossible for each person to remember.  A simpler policy might be that they all need a 2-pass authentication scheme (4-character pin and 8+ character password).  The training is another key part of the ESA.  All stakeholders would have to receive training, including CCU employees, credit union employees, and the users themselves would need some sort of training (not necessarily formal).  

\section{Responsibility}
The CEO should be ultimately responsible, as he/she would ideally be the sponsor of the whole project.  However, the responisbility also falls on the Security Officer (Linda or a new hire), as well as the violator of the policies in part.  

\section{Critical Success Factors & Priorities}
Critical factors for the success of an ESA starts with the sponsor, the person who is ultimately responsible must be a high level executive that can actually get things done.  The next is structuring the policies and standards correctly, so that there will be acceptance in all areas of the company and stakeholders.  After creating the organizational structure, there needs to be risk assessment, baseline testing to compare against after time periods to benchmark the success of the ESA.

\end{document}

