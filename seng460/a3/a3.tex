\documentclass{article}
\usepackage[utf8]{inputenc}
\usepackage[T1]{fontenc} % Use 8-bit encoding that has 256 glyphs
\usepackage{fourier} % Use the Adobe Utopia font for the document - comment this line to return to the LaTeX default
\usepackage[english]{babel} % English language/hyphenation
\usepackage{amsmath,amsfonts,amsthm} % Math packages
\usepackage{changepage}
\usepackage{sectsty} % Allows customizing section commands
\allsectionsfont{\centering \normalfont\scshape} % Make all sections centered, the default font and small caps

\usepackage{fancyhdr} % Custom headers and footers
\pagestyle{fancyplain} % Makes all pages in the document conform to the custom headers and footers
\fancyhead{} % No page header - if you want one, create it in the same way as the footers below
\fancyfoot[L]{} % Empty left footer
\fancyfoot[C]{} % Empty center footer
\fancyfoot[R]{\thepage} % Page numbering for right footer
\renewcommand{\headrulewidth}{0pt} % Remove header underlines
\renewcommand{\footrulewidth}{0pt} % Remove footer underlines
\setlength{\headheight}{13.6pt} % Customize the height of the header

\numberwithin{equation}{section} % Number equations within sections (i.e. 1.1, 1.2, 2.1, 2.2 instead of 1, 2, 3, 4)
\numberwithin{figure}{section} % Number figures within sections (i.e. 1.1, 1.2, 2.1, 2.2 instead of 1, 2, 3, 4)
\numberwithin{table}{section} % Number tables within sections (i.e. 1.1, 1.2, 2.1, 2.2 instead of 1, 2, 3, 4)

\setlength\parindent{0pt} % Removes all indentation from paragraphs - comment this line for an assignment with lots of text

%----------------------------------------------------------------------------------------
%	TITLE SECTION
%----------------------------------------------------------------------------------------

\newcommand{\horrule}[1]{\rule{\linewidth}{#1}} % Create horizontal rule command with 1 argument of height

\title{
\large{\textsc{University of Victoria Software Engineering}}\huge\\ [0pt] % Your university, school and/or department name(s)
\horrule{0.5pt}\\[0.4cm]
\textsc{Case Study : Develop a Security Threat Risk Assessment for UVic IT}\\
\author{Braden Simpson\\braden@uvic.ca\\V00685500}
\date{January 24, 2013}
}

\begin{document}

\maketitle % Print the title

%----------------------------------------------------------------------------------------
%	INTRO
%----------------------------------------------------------------------------------------

\section{Introduction}
\label{sec:intro}
This report is a consultation for the the University of Victoria IT department, to desribe a method of creating an efficient and effective Security Threat Risk Assessment (STRA).  The proposed STRA will go through the following: Preparation Phase, Asset Identification Phase, and the Threat Assessment Phase, and it will cover the university's web environment and services.  The UVic web environment has "...many apparent weaknesses in [the] system..."\cite{caseStudy}  The STRA will assess the most valuable assets to secure, and the most important weaknesses to attend to.

\section{Preparation Phase}
The Preparation phase of the STRA starts off the project, and therefore is crucial to it's success.  The following are the deliverables for the Preparation Phase: 

\begin{itemize}
	\item Define roles for all members involved in the STRA - Have succint responsibilities written up and have them assigned early.  
	\item Analyze the current security practices in th place, and any other policies or standards being used. 
	\item Identify current threats and vulnerabilities that have been reported, or current problems that have occurred.
	\item Identify all stakeholders in the STRA - For example, these could be students, employees, anonymous users visiting the sites, Faculty, IT users, Management, representatives of the university
	\item Define the goals or objectives of the STRA, and set timeframes.
\end{itemize}

Once all of this is done, the preparation phase is finished, and the STRA can move into the Asset Identification Phase.

\section{Asset Identification Phase}
The following table is made to identify the value of certain assets at the University's web environment.  

\begin{center}
    \begin{adjustwidth}{-1.1cm}{}
	    \begin{tabular}{ l | l | l |l  | l | l | l }
	    \hline
	    Class & Category & Group & Confid. & Avail Int & Avail op & Integrity \\
	    \hline\hline
	    Tangible & Information & Univ IT Dept & High & High & High & High \\
	    Tangible & Firmware & Univ IT Dept & & Medium & Medium & Medium \\
	    Tangible & Facilities & Computer Store & & Low & Low &  \\
	    Tangible & Hardware & Univ IT Dept &  & Medium & Medium & High \\
	    Tangible & Software & Univ IT Dept & & Medium & Medium &  \\
	    Tangible & Facilities & Campus Computers & & Medium & Medium &  \\
	    Intangible & University & Reputation & Low & High & High & High \\
	    People & Employees & Univ IT Dept & & High & High & \\
	    People & Employees & Univ Staff & & Medium & Medium & \\
	    People & University & Students & & Medium & Medium & \\
	    People & University & Professors & Medium & Medium & Medium & \\
	    \hline
	    \end{tabular}
    \end{adjustwidth}
\end{center}

The asset values listed in the previous table are a start for the STRA, but when the actual project is underway, the team should investigate further and produce breakdowns for each row listed.  One of the most important assets is the university's reputation, which is shown in the table.

\section{Threat Assessment Phase}
The Threat Assessment Phase is done by completing a table like the following table, in which the likelihood and gravity of threats are combined to create a threat level, which is applied to any of the aspects of Confidentiality, Integrity, and Availability.  The threat level is computed using the tables provided in the case study\cite{caseStudy}
\small
\begin{center}
\label{tab:tap}
\begin{adjustwidth}{-4.3cm}{}
    \begin{tabular}{ l | l | l | l | l | l | l | l | l} 
    \hline
    ID No. & Class & Agent & Event & Likelihood & Gravity & Confid. & Avail. & Integrity \\
    \hline\hline
    31 & Deliberate & Individuals & Network Exploitation & Medium & High &  & Medium &  \\ 
    32 & Deliberate & Individuals & Social Engineering & High & High & High & High & High \\
    40 & Deliberate & Groups/Individuals & Delete/Destroy Records & Medium & Medium &  &  & Medium \\
    41 & Deliberate & Groups/Individuals & Corrupt Data & Medium & Medium &  &  & Medium \\
    42 & Deliberate & Groups/Individuals & Encrypt Files & Medium & Medium &  & Medium &  \\
    43 & Deliberate & Groups/Individuals & Misconfigure Software & High & Medium & High & High & High \\
    44 & Deliberate & Groups/Individuals & Misconfigure Hardware & Medium & High &  & High &  \\
    46 & Deliberate & Wannabees & DOS Attack & Medium & Medium &  & Medium &  \\
    47 & Deliberate & Wannabees & Malicious Code & Low & High & Medium & Medium & Medium \\
    48 & Deliberate & Wannabees & File Corruption & Low & Medium &  &  & Low \\
    60 & Deliberate & Script Kiddies & Web Defacement & Low & Medium & Low & Low & Very Low \\
    94 & Deliberate & Hackers & Identity Theft & Medium & High & High &  &  \\
    103 & Deliberate & Companies & Patent Infringement & Low & Medium & Low &  &  \\
    106 & Deliberate & Individuals & Spam & High & Low & Medium &  &  \\
    108 & Deliberate & Individuals & Unauthorized Use & Medium & Medium & Medium & Medium & Medium \\
    118 & Accidents & Individuals & Inaccurate Data Input & High & Low &  &  & Medium \\
    121 & Accidents & Office Staff & Delete Files & High & Low &  &  & Medium \\
    122 & Accidents & Office Staff & Spill Liquids & Low & Low &  & Very Low & Very Low \\
    126 & Accidents & Cleaning Staff & Unplug Equipment & Medium & Low &  & Low &  \\
    127 & Accidents & Individuals & Lose Laptop & High & High & Very High &  &  \\
    129 & Accidents & Data Entry Clerks & Data Entry Errors & High & Low &  &  & Medium \\
    130 & Accidents & Database Admin & Operating Errors & High & Low &  & Medium & Medium \\
    131 & Accidents & Companies & Software Bugs & High & Medium & High &  & High \\
    132 & Accidents & Organizations & Software Integration Errors & High & Medium &  & High &  \\
    133 & Accidents & Individuals & Coding Errors & High & Low &  &  & Medium \\
    134 & Accidents & Individuals & Software Configuration Errors & High & Low &  & Medium &  \\
    135 & Accidents & Companies & Design Flaws & High & Medium & High &  & High \\
    136 & Accidents & Companies & Equipment Malfunction & Medium & Medium &  & Medium &  \\
    137 & Accidents & Organizations & Installation Errors & Medium & Low &  & Low &  \\
    138 & Accidents & Individuals & Hardware Configuration Errors & Medium & Low &  & Low &  \\
    139 & Accidents & Individuals & Operator Errors & High & Low &  & Medium & Medium \\
    147 & Accidents & Individuals & Inadvertent Misuse & High & Low &  & Medium & Medium \\
    156 & Accidents & Equipment Operators & Disrupt Production & High & Medium &  & High &  \\
    208 & Natural Hazards & Dust & Media Contamination & Low & Low &  & Very Low & Very Low \\
    \hline
    \end{tabular}
\end{adjustwidth}
\end{center}
\normalsize
The UVic IT Department would need to make a more complete version of the above table, and then continue on to perform vulnerability assessments, residual risk assessments and developing final recommendations for the IT department.
\bibliographystyle{IEEEtran} 
\bibliography{a3}
\end{document}