% Braden Simpson
% Lab 2, Astronomy 101
\documentclass{article}
\usepackage{graphicx}
\usepackage{amsmath}
\usepackage{fixltx2e}
% Correct bad hyphenation here
\hyphenation{op-tical net-works semi-conduc-tor}

% Begin the paper here
\begin{document}

% Paper title
\title{Lunar Imaging \\ Lab 3 \\ Astronomy 101}
% Authors names
\author{
Braden Simpson \\ braden@uvic.ca \\ V00685500
}
% make the title area
\maketitle

\section{Objective}
To observe portions of the moon through images and interpret the visuals and craters by using data to locate notable areas.
\section{Introduction}
This lab includes a large amount of history about the moon, how the lunar knowledge was gathered, and other notable points about the moon.  It will include portions about the moon's phases, eclipses, Mare, craters, and the excercises will include lunar charts and historical data about notable areas of the moon.  
\section{Equipment}
\begin{itemize}
\item Moon Map
\item Large cutout moon pictures
\item Ruler
\item Calculator
\end{itemize}
\section{Procedure}
\subsection{Constructing Moon Images}
Cut out the sections of the moon and paste them together to form the larger moon picture.  Then using the data provided in the lab report, answered the questions in the excercises.  This includes finding craters on the 
moon maps, calculating asteroid sizes, and calculating if a tsunami caused by an asteroid will reach parts of victoria.  

%---------------------- EXERCISES -----------------------%

\subsection{Exercises}
2) Identify three Maria, three landing sites, and three craters.  Note if there are central peaks in the craters.  The numbers denote the corresponding number on the constructed moon map.
\begin{enumerate}
	\item{Mare Crisium}
	\item{Mare Nectaris}
	\item{Mare Fecunditatis}
	\item{A11}
	\item{A17}
	\item{A16}
	\item{134 Theophilus -- Yes, central peaks present}
	\item{169 Santbech -- No, no central peaks}
	\item{198 Piccolomini -- Yes, central peaks present}
\end{enumerate}
3) Find and indicate an example of a crater, which was formed
after another crater. We know that the dark Maria are about 3.5 Billion
years old. Find a crater which formed after this. Find one that formed
before 3.5 Billion years ago.\\
\textbf{Before}\\
\emph{84 Julius Ceasar} -- The crater looks as though it has been filled in over time and therefore means that it must have been created before 3.5 billion years ago when the dark Maria formed.\\
\textbf{After}\\
\emph{78 Picard} -- The crater appears inside the dark mare, meaning it must have formed after they did.\\
\textbf{Can you find a crater that is lighter in color?} -- Yes, one example of such a crater is \emph{140 -- Langrenus}\\
\textbf{Can you find a crater that has lines of lighter color eminating from the crater?} -- Yes, one crater is \emph{99 -- Taruntius} with lines going towards Mare Fecunditatus.\\

4) Measure the diameter of a big and a litter crater on the moon and convert them to the kilometers assuming a diameter of the moon of 3476 km.  If the diameter of a crater is about 10 to 50 times (average 25) times the diameter of the meteorite depending on the mass and velocity of the meteorite.  How big is the meteorite that made the craters chosen, assuming craters are 25 times the size of the meteorite?

\begin{equation}
d_crater  = 
\end{equation}

\subsection{Geocentric Ecliptic Longitude of the Planets}
In this section, instead of plotting the planets heliocentrically, they will be plotted using the geocentric ecliptic longitude, meaning that the results will be relative to the earth.  This will give the approximate constellation, right ascension, and declination of the planets, allowing the planets to be found from earth.  After getting the geocentric ecliptic longitude from the polar coordinate map, then they can be plotted on the SC001 Constellation chart.
\subsection{Use of a Planisphere}
A planisphere can plot the stars in the sky, based on time of day and date of the year.  The visible sky is shown, with the constellations positioned as they would be at the date. 
\section{Observations}
See attached constellation chart and polar coordinates map. 
\section{Measurements}
See Tables 1 and 2, attached constellation chart, and polar coordinate map.
\section{Questions}
\begin{enumerate}
\item Which star stays in the visible area of the planisphere at all times?\\
\emph{Polaris}\\ 
\item Turn dial until 12 matches up with 01 July.  Star in the middle part of the dial will be one passing overhead in the Zenith.  What is the name of the star in the Zenith?  What Constellation is it in?\\
\emph{Polaris, from Ursa Minor}\\
\item On the right hand side of the planisphere is the "Western Horizon".  Which star is on the Western Horizon?\\
\emph{Spica}\\
\item Turn to 15 July at 11pm.  Which star is in the Zenith?  Which star is on the Western Horizon?\\
\emph{Polaris, and Spica}
\item Turn the dial to 15 August at 11pm.  Which star is on the Zenith?  Which star is on the Western Horizon?\\
\emph{Deneb from Cygnus, Antares is on the Western Horizon.}
\item What time will Antares set on the 22nd of September?  What day and month will Antares set at 1am?\\
\emph{9:22pm on September 22nd.  Antares will set at 1am on July 22nd.}\\
\item Turn dial to 11pm on January 01.  Find Sirius.  What time will Sirius rise on January 01?  What time will sirius set on January 01?  For how many hours will Sirius be on the horizon during 01 January?\\
\emph{6:30pm Rise.  5:00am Set.  Total of 10.5 hours in the sky.}
\end{enumerate}
\section{Conclusion}
The planets lab was a success, I have learned about how the planets move by using both heliocentric, and geocentric coordinates.  The lab included multiple chances for hand plotting points and using real data to determine actual positions of the planets and constellations.  This gave a realistic useful experience, and the techniques learned can be reused for future experiments. 
\begin{table}
\begin{center}
\begin{tabular}{@{\hspace{.2cm}}ccc@{\hspace{.2cm}}c@{\hspace{.2cm}}c@{\hspace{.2cm}}c@{\hspace{.2cm}}c@{\hspace{.2cm}}}
\hline
Planet & Noon & Sunset & Midnight & Sunrise & Configuration \\
\hline
Venus	&	Y&	N&	N&	Y&	EW\\
Mars		&	Y&	Y&	N&	N&	Op/Q\\
Jupiter	&	N&	N&	Y&	Y&	Op\\
Saturn	&	Y&	Y&	N&	N&	C\\
\hline
\end{tabular}
\end{center}
\caption{Table of the planets as seen from Earth\label{tab:ratio}}
\end{table}
\begin{table}
\begin{center}
\begin{tabular}{@{\hspace{.2cm}}ccc@{\hspace{.2cm}}c@{\hspace{.2cm}}c@{\hspace{.2cm}}c@{\hspace{.2cm}}c@{\hspace{.2cm}}}
\hline
Planet & Ecliptic Long. (degrees) & Constellation & Right Ascension & Declination (degrees)\\
\hline
Sun		&	180&	Virgo&	12h&			0\\
Venus	&	136&	Cancer&	9h15m&		16\\
Mars 	&	233&	Libra&	15h20m&		-19\\
Jupiter	&	201&	Virgo&	3h 15m&		-9\\
Saturn	& 	75&		Taurus&	4h 55m&		22\\
\hline
\end{tabular}
\end{center}
\caption{Geocentric Equatorial Position of the Planets\label{tab:ratio}}
\end{table}
\bibliographystyle{IEEEtran}
\bibliography{lab1}
\end{document}