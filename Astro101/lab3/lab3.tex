% Braden Simpson
% Lab 2, Astronomy 101
\documentclass{article}
\usepackage{graphicx}
\usepackage{amsmath}
\usepackage{fixltx2e}
% Correct bad hyphenation here
\hyphenation{op-tical net-works semi-conduc-tor}

% Begin the paper here
\begin{document}

% Paper title
\title{Lunar Imaging \\ Lab 3 \\ Astronomy 101}
% Authors names
\author{
Braden Simpson \\ braden@uvic.ca \\ V00685500
}
% make the title area
\maketitle

\section{Objective}
To observe portions of the moon through images and interpret the visuals and craters by using data to locate notable areas.
\section{Introduction}
This lab includes a large amount of history about the moon, how the lunar knowledge was gathered, and other notable points about the moon.  It will include portions about the moon's phases, eclipses, Mare, craters, and the excercises will include lunar charts and historical data about notable areas of the moon.  
\section{Equipment}
\begin{itemize}
\item Moon Map
\item Large cutout moon pictures
\item Ruler
\item Calculator
\end{itemize}
\section{Procedure}
\subsection{Constructing Moon Images}
Cut out the sections of the moon and paste them together to form the larger moon picture.  Then using the data provided in the lab report, answered the questions in the excercises.  This includes finding craters on the 
moon maps, calculating asteroid sizes, and calculating if a tsunami caused by an asteroid will reach parts of victoria.  

%---------------------- EXERCISES -----------------------%

\subsection{Exercises}
2) Identify three Maria, three landing sites, and three craters.  Note if there are central peaks in the craters.  The numbers denote the corresponding number on the constructed moon map.
\begin{enumerate}
	\item{Mare Crisium}
	\item{Mare Nectaris}
	\item{Mare Fecunditatis}
	\item{A11}
	\item{A17}
	\item{A16}
	\item{134 Theophilus -- Yes, central peaks present}
	\item{169 Santbech -- No, no central peaks}
	\item{198 Piccolomini -- Yes, central peaks present}
\end{enumerate}
3) Find and indicate an example of a crater, which was formed
after another crater. We know that the dark Maria are about 3.5 Billion
years old. Find a crater which formed after this. Find one that formed
before 3.5 Billion years ago.\\
\textbf{Before}\\
\emph{84 Julius Ceasar} -- The crater looks as though it has been filled in over time and therefore means that it must have been created before 3.5 billion years ago when the dark Maria formed.\\
\textbf{After}\\
\emph{78 Picard} -- The crater appears inside the dark mare, meaning it must have formed after they did.\\
\textbf{Can you find a crater that is lighter in color?} -- Yes, one example of such a crater is \emph{140 -- Langrenus}\\
\textbf{Can you find a crater that has lines of lighter color eminating from the crater?} -- Yes, one crater is \emph{99 -- Taruntius} with lines going towards Mare Fecunditatus.\\
\\
4) Measure the diameter of a big and a litter crater on the moon and convert them to the kilometers assuming a diameter of the moon of 3476 km.  If the diameter of a crater is about 10 to 50 times (average 25) times the diameter of the meteorite depending on the mass and velocity of the meteorite.  How big is the meteorite that made the craters chosen, assuming craters are 25 times the size of the meteorite?\\

Crater 140
\begin{equation}
d_{crater}  = \frac{38.5cm}{3476km} = \frac{1.5cm}{xkm} = 135.43km
\end{equation}
\begin{equation}
d_{meteor}  = \frac{d_{crater}}{25} = 5.41km
\end{equation}

Crater 110
\begin{equation}
d_{crater}  = \frac{38.5cm}{3476km} = \frac{.3cm}{xkm} = 27.1km
\end{equation}
\begin{equation}
d_{meteor}  = \frac{d_{crater}}{25} = 1.1km
\end{equation}\\
\\
5) In the Yucatan peninsula a crater has now been identi�ed which is
180 km in diameter and 65 million years old. Is this crater about the
right size? \\
\begin{equation}
km_{avg} = 10 * 25 = 250
\end{equation}
\begin{equation}
km_{min} = 10 * 10 = 100
\end{equation}
\begin{equation}
km_{max} = 10 * 50 = 500
\end{equation}
\begin{equation}
km_{min} < km_{crater} < km_{max} = 100km < 180km < 500km
\end{equation}

Therefore by 8) the crater in the Yucatan peninsula is a crater within the reasonable range.\\
\\ 
6) It has been estimated that most of the people on the earth would be
killed by the impact of a much smaller 1 km. diameter object. This is the equivalent of a 1 million megaton explosion and would make a
crater about 25 km. across. To estimate how often this happens we
can examine our nearest neighbour, the moon. There are 29�5 craters
bigger than 25 km in diameter on the lunar Maria. The maria are 3.5
billion years old, so how often do craters of this size form on the maria?
To estimate how often this happens on the earth we need to know how
much bigger the earth is compared to the lunar maria. If the area of
the maria is about 5 million sq. km, and if the area of the earth is 500
million sq. km, how often would you expect a crater of this size to be
formed on the earth and end civilization?\\

\begin{equation}
d_{Earth} = 100 * d_{Moon}
\end{equation}
Therefore the number of asteroids that will hit Earth over 3.5 billion years are given by assuming 29 is the average amount of asteroids 29 plus or minus 5
\begin{equation}
asteroids  = 29 * 100
\end{equation}
By 10) Asteroids / years is given by
\begin{equation}
A_{year} = \frac{2900}{3,500,000,000} = 8.29 *10^{-8}
\end{equation}
Therefore there are 8 asteroids of that size every 10 million years.\\
The lower bound, using 24 as asteroids is given by 
\begin{equation}
lower = \frac{24}{3,500,000,000} * 100 = \frac{685.7}{1,000,000,000 years}
\end{equation}
\\
The upper bound, using 34 as asteroids is given by 
\begin{equation}
upper = \frac{34}{3,500,000,000} * 100 = \frac{971.4}{1,000,000,000 years}
\end{equation}
\\
7) Every year on the 12 August the Perseid Meteor shower occurs. This
meteor shower is caused by bits of gravel lost from comet Swift-Tuttle,
but still traveling in almost the same orbit. The Earth�s orbit intersects
the comet�s orbit at the place that the Earth is on 12 August. If the
date of perihelion of the periodic comet Swift-Tuttle changes by +15
days (it changed by several years in its last orbit), it will hit the earth
on August 14, 2126. It has a diameter of about 2 km, and would hit the
earth with a speed of about 50 km/s. How big a crater would it make?
There would be a 75\% chance that it would land in an ocean. From the
table, how large would the tsunami be 300 km from the impact sight?
Would you be safe on Mt. Doug (altitude=210 m)?\\
From the table given in Figure x, we can determine that since we are 210m up, and the impact site is 300km away, creating a Tsunami height = 200m, we would be safe on Mt. Doug.

\section{Tables and Measurements}
\label{sec:tnm}

\begin{table}[h]
\begin{center}
\begin{tabular}{c c c c}
\hline
\hline
Crater Diameter (km) & 150 & 50 & 10\\
\hline
Energy (ergs) & 1 X 10$^{30}$ & 6 X 10$^{28}$ & 2.6 X 10$^{26}$\\
\hline
\hline
Distance (km) & Tsunami Height (meters) &  & \\
10000 & 100m & 15m & 1m\\
3000 & 250m & 40m & 3m\\
1000 & 540m & 90m & 5m\\
300 & 1300m & 200m & 15m\\
\hline
\end{tabular}
\end{center}
\caption{Crater and Tsunami Size for Various Impact Energies.\label{tab:waves}}
\end{table}

\begin{table}[h]
\begin{center}
\begin{tabular}{l c c c}
\hline
Crater Size & Scale Size (mm) & Real Size (km) & Meteorite Real Size (km)\\
\hline
\hline
Small & 3mm & 27.09km & 1.08km\\
Large & 15mm & 135.4km & 5.41km\\
\hline
\end{tabular}
\end{center}
\caption{Measurements of impact craters on the moon.\label{tab:craters}}
\end{table}
\end{document}