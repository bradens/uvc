% Braden Simpson
% Lab 1, Astronomy 101
\documentclass{article}
\usepackage{graphicx}
\usepackage{amsmath}
\usepackage{fixltx2e}
% Correct bad hyphenation here
\hyphenation{op-tical net-works semi-conduc-tor}

% Begin the paper here
\begin{document}

% Paper title
\title{Night Sky \\ Lab 1 \\ Astronomy 101}
% Authors names
\author{
Braden Simpson \\ braden@uvic.ca \\ V00685500
}
% make the title area
\maketitle

\section{Objective}
To learn the essentials of astronomy -- planets, stars, galaxies, nebulae, and telescopes -- through observation of the night sky.
\section{Equipment}
\subsection{Telescopes}
Reflecting Cassegraine Telescope, Refracting Telescope, see the attached equipment page for diagrams.\\
Primary Mirror -- \emph{Large curved mirror that reflects light into the secondary mirror} \\
Secondary Mirror -- \emph{Reflects the light for the second time, into the eyepiece} \\
Focuser -- \emph{Knob used to adjust the focus of the telescope} \\
Mount -- \emph{Used to keep the telescope mounted in correct position} \\
Finder -- \emph{Refractive telescope to find the general area to narrow in on before using the main telescope}
\subsection{Brightness}
The brightness factor is calculated by taking the ratio of the area of the mirror A\textsubscript{m} divided by the area of the pupil, A\textsubscript{p}, defined below.
\begin{equation}
\frac{\text{A}_{m}}{\text{A}_{p}} = \frac{\pi (10cm)^2}{\pi (0.5cm)^2} = 400
\end{equation}
Therefore the brightness multiplier for the 8inch reflective telescope is 400, assuming 20cm for the diameter of the primary mirror, and 1cm diameter for the human pupil. \\
Similarly, the magnification can be calculated by the ratio of the telescope's focal length (2000mm in our case) divided by the eyepiece's focal length (40mm).
\begin{equation}
\text{M} = \frac{\text{f}_{primary}}{\text{f}_{eyepiece}} = \frac{2000}{40} = 50
\end{equation}
Yielding a 50 times magnification.

\section{Observations}
\subsection{Constellations}
See attached constellation diagrams.

\subsubsection{Andromeda}
Andromeda is located north of the celestial equator, and is only visible north of 40 degrees south latitude. \cite{andromedastarmap} It is a very large constellation an area of 722 square degrees.
\paragraph{Mythology:}
Andromeda was the beautiful daughter of Cassiopeia, who was chained to a rock in the sea by Poseidon, to be eaten by the sea monster Cetus.  This happened because Andromeda's father, Cepheus was the king of Aethiopia, and the only way to save his kingdom would be to sacrifice his daughter.  The daughter was saved by Perseus, who wielded the head of the medusa and turned the monster Cetus to stone.
\paragraph{Stars}
Andromeda is interesting because it's brightest star \emph{Alpheratz} with a magnitude of 2.1 and a distance of 97 light-years from Earth, is also the beginning of another constellation, \emph{Pegasus}.  There are also two more notable stars (there are more but I was only able to easily discern three when drawing it during the lab), \emph{Mirach}, and \emph{Almach}.

\subsubsection{Cepheus}
Located near Cassiopeia, and Polaris, Cepheus is shaped somewhat like a sideways child's drawing of a house.
\paragraph{Mythology}
Cepheus was the king of Aethiopia, father of Andromeda, and husband of Cassiopeia.  Cepheus offered his daughter as a sacrifice to save his kingdom, but was always praying for her to live. After Perseus came to save Andromeda from the sea monster Cetus, he allowed Perseus to marry Andromeda instead of his brother Phineus. \cite{cepheusmyth}
\subsubsection{Perseus}
\paragraph{Mythology}
Perseus is the hero who slain the gorgon Medusa and carries her head.  He used the Medusa head to turn the monster Cetus to stone. \cite{perseusmyth} 
\paragraph{Stars}
Perseus is home to \emph{Mirfak} ($\alpha$ Per), the brightest star, which is a supergiant, with luminosity 5,000 times and it's diameter 42 times that of our sun. \\ \cite{perseusstarmap}
Perseus also contains the star \emph{Algol} ($\beta$ Per), which is also known as "Demon Head", because it is the eye of the gorgon Medusa.  It is approximately 92.8 light-years from Earth.  One of the interesting parts of Algol is that is a variable magnitude star from a minimum of 3.5 to a maximum of 2.3, with a period of 2.867 days. \cite{perseus}
\subsubsection{Aquila}
Located in the northern sky, it lies a few degrees north of the celestial equator.  Altair, the constellation's brightest star is a vertex of the Summer Triangle asterism.
\paragraph{Mythology}
Aquila is believed to have been the bird that carried Zues' thunderbolts.  It is also said that Aquila is the eagle who kidnapped Ganymede. 
\paragraph{Stars}
The three main stars of Aquila are \emph{Altair}, \emph{Alshain}, and \emph{Tarazed}
\subsection{The Stars}
\subsubsection{Albireo}
Albireo is interesting because it is a double star, Albireo A, which is red and a magnitude of 3.1, and Albireo B, blue and magnitude 5.1.  Albireo A, burns cooler (approximately 4000K at its hottest whereas B burns at approximately 13000K) 
\subsection{Deep Sky Objects}
Sketches can be found attached in pen.
\subsubsection{Globular Cluster}
\paragraph{Messier 15 (M15)}
A globular cluster with approximately 100,000 stars, approximately 33,000 light-years away and 12 billion years old.  M15 has a supermassive black hole in the center, which holds the stars in, in a tight gravitational pull.
\subsubsection{Open Cluster}
\paragraph{Messier 11 (M11)}
Open cluster with approximately 3000 stars, which are all similar in age, since they were formed from one very large clump of gas approximately 1000 light years from Earth.  Open clusters are loosely bound by mutual gravitational pull, which is different from \emph{Globular Clusters}.
\subsubsection{Planetary Nebula}
A planetary nebula consists of expanding gas.  They are essential to the chemical evolution of the galaxy, meaning they are good for returning materials that have been enriched back to the galaxy.
\paragraph{M57 -- Ring Nebula}
M57 is a great example of a planetary nebula because it contains a now white dwarf , that had gone supernova and exploded, sending gas expanding back into the galaxy.  It is approximately 2300 light years away.\cite{ringnebula}    
\subsubsection{Galaxy}
A galaxy is a large grouping of stars, gas, dust, and other objects that are bound together by gravity.  There are estimated to be more than 170 billion galaxies in the observable universe, one of which is our Milky Way galaxy.
\paragraph{Andromeda -- M31}
Andromeda is another spiral galaxy nearest to our own, at an approximate distance of 2.5 million light years away.  The interesting thing about the Andromeda galaxy is that it is one of the brightest spiral galaxies, making it slightly visible to the naked eye on some nights.  
\section{Conclusion}
This lab was a success in that I know a lot more about many new constellations including the four included in the report as well as more such as Signus, and Saggitarius.  I also have knowledge about two types of telescope, reflective and refractive, and the uses of both, in addition to being able to operate one.  
\bibliographystyle{IEEEtran}
\bibliography{lab1}
\end{document}