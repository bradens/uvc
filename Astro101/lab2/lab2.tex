% Braden Simpson
% Lab 2, Astronomy 101
\documentclass{article}
\usepackage{graphicx}
\usepackage{amsmath}
\usepackage{fixltx2e}
% Correct bad hyphenation here
\hyphenation{op-tical net-works semi-conduc-tor}

% Begin the paper here
\begin{document}

% Paper title
\title{Positions of the Planets \\ Lab 2 \\ Astronomy 101}
% Authors names
\author{
Braden Simpson \\ braden@uvic.ca \\ V00685500
}
% make the title area
\maketitle

\section{Objective}
This lab’s purpose is to educate the student with the planets, how they move relative to each other, how their position plays a role in what planets can be seen, and what the configurations of planets are.  Finally the student will be able to make predictions about where planets will be in the sky.
\section{Introduction}
In this lab, there will be plotting of planets on coordinate and constellation maps, based on heliocentric values.  These plottings will then be used to give orbits and semi-accurately predict where the planets will be at times in the year.
\section{Equipment}
\begin{itemize}
\item Large sheet of Polar coordinate graph paper
\item Protractor
\item SC001 Constellation chart
\item Coloured pencils
\item Planisphere
\end{itemize}
\section{Procedure}
\subsection{Plotting Planets}
The planets data was given in heliocentric coordinates, for given times of the year.  From those coordinates, the planets were plotted on the Polar coordinate graph paper, using coloured pencils to differentiate between the dates.  The data was given in the form of two tables, one which had the heliocentric longitudes of the planets, and another with the Astronomical Unit distance of the planets.  In order to do this plotting, the orbits were assumed to be circular. 
\subsection{Conjunctions, Elongations, and Oppositions}
In this section, the planets that have been plotted will have their configuration determined, and whether or not they are visible at sunrise, sunset, noon, and midnight.  This can all be determined based on their position relative to the earth and sun.
\subsection{Geocentric Ecliptic Longitude of the Planets}
In this section, instead of plotting the planets heliocentrically, they will be plotted using the geocentric ecliptic longitude, meaning that the results will be relative to the earth.  This will give the approximate constellation, right ascension, and declination of the planets, allowing the planets to be found from earth.  After getting the geocentric ecliptic longitude from the polar coordinate map, then they can be plotted on the SC001 Constellation chart.
\subsection{Use of a Planisphere}
A planisphere can plot the stars in the sky, based on time of day and date of the year.  The visible sky is shown, with the constellations positioned as they would be at the date. 
\section{Observations}
See attached constellation chart and polar coordinates map. 
\section{Measurements}
See Tables 1 and 2, attached constellation chart, and polar coordinate map.
\section{Questions}
\begin{enumerate}
\item Which star stays in the visible area of the planisphere at all times?\\
\emph{Polaris}\\ 
\item Turn dial until 12 matches up with 01 July.  Star in the middle part of the dial will be one passing overhead in the Zenith.  What is the name of the star in the Zenith?  What Constellation is it in?\\
\emph{Polaris, from Ursa Minor}\\
\item On the right hand side of the planisphere is the "Western Horizon".  Which star is on the Western Horizon?\\
\emph{Spica}\\
\item Turn to 15 July at 11pm.  Which star is in the Zenith?  Which star is on the Western Horizon?\\
\emph{Polaris, and Spica}
\item Turn the dial to 15 August at 11pm.  Which star is on the Zenith?  Which star is on the Western Horizon?\\
\emph{Deneb from Cygnus, Antares is on the Western Horizon.}
\item What time will Antares set on the 22nd of September?  What day and month will Antares set at 1am?\\
\emph{9:22pm on September 22nd.  Antares will set at 1am on July 22nd.}\\
\item Turn dial to 11pm on January 01.  Find Sirius.  What time will Sirius rise on January 01?  What time will sirius set on January 01?  For how many hours will Sirius be on the horizon during 01 January?\\
\emph{6:30pm Rise.  5:00am Set.  Total of 10.5 hours in the sky.}
\end{enumerate}
\section{Conclusion}
The planets lab was a success, I have learned about how the planets move by using both heliocentric, and geocentric coordinates.  The lab included multiple chances for hand plotting points and using real data to determine actual positions of the planets and constellations.  This gave a realistic useful experience, and the techniques learned can be reused for future experiments. 
\begin{table}
\begin{center}
\begin{tabular}{@{\hspace{.2cm}}ccc@{\hspace{.2cm}}c@{\hspace{.2cm}}c@{\hspace{.2cm}}c@{\hspace{.2cm}}c@{\hspace{.2cm}}}
\hline
Planet & Noon & Sunset & Midnight & Sunrise & Configuration \\
\hline
Venus	&	Y&	N&	N&	Y&	EW\\
Mars		&	Y&	Y&	N&	N&	Op/Q\\
Jupiter	&	N&	N&	Y&	Y&	Op\\
Saturn	&	Y&	Y&	N&	N&	C\\
\hline
\end{tabular}
\end{center}
\caption{Table of the planets as seen from Earth\label{tab:ratio}}
\end{table}
\begin{table}
\begin{center}
\begin{tabular}{@{\hspace{.2cm}}ccc@{\hspace{.2cm}}c@{\hspace{.2cm}}c@{\hspace{.2cm}}c@{\hspace{.2cm}}c@{\hspace{.2cm}}}
\hline
Planet & Ecliptic Long. (degrees) & Constellation & Right Ascension & Declination (degrees)\\
\hline
Sun		&	180&	Virgo&	12h&			0\\
Venus	&	136&	Cancer&	9h15m&		16\\
Mars 	&	233&	Libra&	15h20m&		-19\\
Jupiter	&	201&	Virgo&	3h 15m&		-9\\
Saturn	& 	75&		Taurus&	4h 55m&		22\\
\hline
\end{tabular}
\end{center}
\caption{Geocentric Equatorial Position of the Planets\label{tab:ratio}}
\end{table}
\bibliographystyle{IEEEtran}
\bibliography{lab1}
\end{document}